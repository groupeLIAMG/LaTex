% !TeX spellcheck = fr_CA
\documentclass[../exemple_master.tex]{subfiles}

\begin{document}

Ceci est un document d'initiation à Latex utilisant la classe de document fournie par Elsevier. L'utilisation de Latex peut permettre de faciliter grandement le processus de soumission des articles à des revues scientifiques et permet de sauver beaucoup de temps pour la mise en page des thèses par article. Par exemple, pour \emph{Journal of Hydrology}, la soumission initiale consiste à soumettre une version pdf de l'article, avec la mise en page spécifiée dans la classe de document fourni par Elsevier. La seconde étape consiste à soumettre une version pdf révisée de l'article, de même qu'une version mettant en évidence les modifications apportées. Cela peut être fait dans Latex grâce à l'outil \emph{latexdiff} (\url{www.ctan.org/pkg/latexdiff}). La soumission finale consiste tout simplement à produire un fichier \emph{.zip} du dossier Latex incluant les figures et tous les documents Latex. Puisque les images sont déjà séparées du code dans Latex, cela permet de faciliter et d'accélérer le processus de soumission de la version finale de l'article. Cet exemple a été compilé avec succès avec la distribution \emph{TexLive}~(\url{www.tug.org/texlive}) en utilisant le compilateur \emph{pdflatex}.

\end{document}