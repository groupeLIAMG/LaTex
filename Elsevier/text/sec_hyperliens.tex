\documentclass[../exemple_master.tex]{subfiles}

\begin{document}

\section{Les hyperliens}

Grâce au module \emph{cref}, le mot "figure" n'a pas besoin d'être mis dans le code. \emph{cref} se charge de cela automatiquement. Par exemple, voici des hyperliens vers la \cref{fig1_exemple} et vers le \cref{tbl_mini_exemple}. On peut également faire des références multiple automatiquement. Par exemple, on pourrait référer aux \cref{fig1_exemple,fig2_exemple} avec une seule commande grâce à \emph{cref}.

C'est pratique si on voulait plus tard changer le format des labels, par exemple, écrire figure 1 ou Fig. 1 ou fig.1 au lieu de Figure 1. On aurait alors qu'une option à changer dans le préambule du document lorsque l'on charge le module \emph{cref} et le style de toutes les références serait mis à jour dans le document. Cela permet d'assurer une uniformité complète du style dans tout le document et évite les erreurs typographiques.

\end{document}