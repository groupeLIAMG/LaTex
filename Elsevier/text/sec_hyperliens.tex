% !TeX spellcheck = fr_CA
\documentclass[../exemple_master.tex]{subfiles}

\begin{document}

\section{Les hyperliens}

Grâce au module \emph{cref}, le libellé des hyperliens (e.g., figure, tableau, équation) n'a pas besoin d'être écrit dans le code. \emph{cref} se charge de cela automatiquement. Par exemple, voici des hyperliens vers la \cref{fig1_exemple} et vers le \cref{tbl_mini_exemple}. Des références multiples peuvent également être incluses dans le document avec une seule commande de code avec \emph{cref}. Par exemple, voici un exemple d'hyperlien multiple vers les \cref{fig1_exemple,fig2_exemple}.

Cette fonctionnalité est pratique pour changer ultérieurement la mise en forme des hypertliens, par exemple: écrire figure 1 ou Fig. 1 ou fig.1 au lieu de Figure 1. Il suffirait de changer qu'une seule option dans le préambule du document, lorsque l'on charge le module \emph{cref}, pour que le style de tous les hyperliens soit mis à jour dans le document. Cela permet d'assurer une uniformité complète du style dans tout le document et d'éviter des erreurs typographiques.

\end{document}