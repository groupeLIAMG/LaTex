\documentclass[exemple_master.tex]{subfiles}

\begin{document}

\section{Les tableaux}

La création des tableaux est problablement une des tâches les plus fastidieuses de Latex. Par contre, le résultat final par contre en vaut la peine. Le \cref{tbl_mini_exemple} présente un example minimal de tableau. Tout comme pour les figures, la position des tableaux est flottante dans le texte.

Le package \emph{siunitx} permet de gérer magnifiquement l'alignement des valeurs numériques. Quelques exemples sont donnés dans le \cref{tbl_mini_exemple}. Un tableau bien construit n'a jamais besoin de lignes verticales. Les colonnes sont définies par un alignement méticuleux du contenu.

% Tout comme pour la figure, on se fait d'abord un floatbox (table) en spécifiant les options de positionnement dans le texte:

\begin{table}[!tbh]
    % On centre tout:
    \centering
    % On définit le caption:
    \caption{On insère le titre du tableau ici}
    % On lui donne un petit nom: 
    \label{tbl_mini_exemple} 
    % On crée un tableau. Chaque lettre dans les {} spécifie un type de colonne.
    \begin{tabular}
    {
     l
     S[table-format = 3.2(2)]
     S[table-format = 1.2e1]
     S[table-format = -2.1]
    }
    % On ajoute une ligne au top du tableau
    \toprule
    % On ajoute du contenu ligne par ligne. Les colonnes sont séparées par des & et les lignes par \\
    Texte & {Incertitudes} & {Scientifiques} & {Décimales} \\
    % On ajoute une ligne pour séparer l'entête du contenu
    \midrule
    Données 1 & 2.41(5) & 2.41e4 & 2.4 \\
    Données 2 & 122.28(15) & 3.13e2 & 12.4 \\
    Données 3 & 12.34(1) & 8.96e5 & -0.4 \\
    % On ajoute une ligne au bas du tableau
    \bottomrule
    \end{tabular}
\end{table}

% Le type des colonnes spécifié entre {} spécifie comment l'alignement des items dans chaque cellule est fait:

% c : centré
% l : justifié à gauche
% r : justifié à droite
% S : Colonne spéciale du package siunitx qui permet l'alignement des chiffres sur la décimale. Il faut spécifier le format des données pour chaque colonne. 

% S[table-format = 3.2(2)] signifie que l'on a 3 chiffres avant et 2 après la décimale, avec une incertitude qui contient 2 décimales.

% S[table-format = 1.2e1] signifie que l'on a 1 chiffres avant et 2 après la décimale, avec une notation décimale avec un exposant à 1 chiffre.

% S[table-format = -2.1] signifie que l'on a 2 chiffres avant et 1 après la décimale et avec des valeurs négatives.

\end{document}