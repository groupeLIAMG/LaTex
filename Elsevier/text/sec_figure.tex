% !TeX spellcheck = fr_CA
\documentclass[../exemple_master.tex]{subfiles}

\begin{document}

\section{Les figures}

La gestion des figures est certainement le principal avantage à utiliser Latex par rapport à MS Word ou Open Office. Les images sont incluses dans le code Latex via un pointeur vers les fichiers images. De cette façon, les images sont incorporées dans le document final uniquement lorsque le code est compilé en pdf. Ainsi, il est possible de mettre toutes les images du document dans un seul dossier. Cela facilite grandement l'étape de soumission des articles aux revues scientifiques, car les fichiers images sont déjà découplés du document. 
La mise à jour des images dans le document est également grandement facilitée puisqu'il suffit simplement de remplacer les fichiers images par les nouvelles versions et de recompiler le code Latex.

Il est possible d'incorporer des images en format pdf, png et jpg avec le module \emph{graphicx}. D'autres modules existent pour les autres formats, mais ils ne sont généralement pas nécessaires: les images vectorielles peuvent être sauvegardées en pdf, ce qui permet de préserver leur qualité dans le document final, alors que les images bitmap peuvent être sauvegardées en png pour les graphiques, logos et schémas et en jpg pour les photos. La \cref{fig1_exemple} présente un exemple de la mise en page Latex d'une image à partir d'un jpg.

Il préférable, autant que possible, d'insérer les figures dans le code à l'endroit où elle devraient à peu près se retrouver dans le texte. Il est généralement nécessaire de jouer un peu avec la position des images dans le code à la fin pour avoir une mise en page optimale. La position des images dans Latex est flottante. C'est-à-dire que Latex va tenter de placer les images à un endroit optimal dans le document en suivant certaines règles que l'on peut spécifier dans le code. Dans l'exemple de la \cref{fig1_exemple}, il a été spécifié de placer l'image autant que possible en haut de la page. S'il s'avérait que cette option soit impossible, Latex placerait l'image en bas de la page, puis à travers le texte ou enfin seule sur une page unique.

Les dimensions des images peuvent également être spécifiées par des valeurs fixes (e.g., \SI{3}{cm}, \SI{55}{mm}, \SI{6}{po}) ou relatives (e.g., \SI{50}{\percent} de la largeur de la page). La seconde option est généralement préférée. Dans l'exemple de la \cref{fig1_exemple}, la largeur de l'image à été spécifiée égale à 50\% de la largeur des colonnes du texte, ce qui permet d'alterner aisément entre une mise en page avec colonne unique et double colonne. Une description détaillée des diverses options pour spécifier des dimensions relatives dans Latex est donné ici: \url{http://tex.stackexchange.com/a/17085/72419}

% Premièrement, on crée un float box (un peu comme les tableaux que tu fais dans Word). Les options entre [] indique comment on veut que le float soit placé au travers du texte: 

% b: en bas de la page ;
% t: au top de la page ;
% h: au-travers du texte ;
% p: tout seul sur une page.
% ! avant une lettre signifie une priorité de cette option si on met plus qu'une option.

\begin{figure}[!tbh]
% On indique d'abord que l'on veut que tout soit centré dans le floatbox.
\centering

% On inclu ensuite la figure dans le floatbox en specifiant le path vers le fichier. Pas besoin de mettre d'extension. Il est possible également de spécifier entre [] des options pour les dimensions de la figure. Dans ce cas, la figure aura une largeur correspondant à 50% de la largeur de la colonne de texte. Il y a d'autre option disponible selon ce que l'on veut faire:
\includegraphics[width=0.5\columnwidth]{img/54701271}

%Ensuite, on spécifie le caption de la figure:
\caption[Petit titre pour la table des matière]{Il est possible de mettre un super méga ultra long titre à la figure et de définir un titre abrégé pour la table des matières, n'est-ce pas merveilleux?}

%Enfin, on lui donne un petit nom de code pour que l'on puisse référer à cette figure dans le texte (voir texte plus haut):
\label{fig1_exemple}
\end{figure}

\end{document}