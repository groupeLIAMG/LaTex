% !TeX spellcheck = fr_CA
\documentclass[../exemple_master.tex]{subfiles}

\begin{document}

\section{Les unités}

Le module \emph{siunitx} permet de gérer les espaces inséparables entre les valeurs numériques et les unités, les espaces dans les grands nombres, l'incertitude, la notation scientifique, le symbole utilisé pour la décimale, la façon dont les suites et les plages de nombres sont affichées, etc. Quelques exemples de valeurs numériques et d'unités affichés grâce au module \emph{siunitx} sont donnés ci-bas: 

\begin{itemize}
\item Unités seules: \si{\percent} ou \si{\celsius} ou \si{\ampere} ou \si{\ohm}
\item Plage de valeurs: \SIrange{25000}{543456743}{W/cm^2 \celsius}
\item Liste de valeurs: \SIlist{25;50;60.7;25.3}{W/cm^2 \celsius}
\item Notation scientifique: \SI{3.04e6}{m^2}
\item Avec incertitude: \SI{3.04(10)}{m^2}
\end{itemize}

On pourrait très bien écrire tout directement dans le code, sans passer par \emph{siunitx}. Toutefois, l'utilisation de \emph{siunitx} permet de s'assurer une uniformité de la mise en forme des valeurs numériques et des unités dans tout le document et permet d'apporter des changements au style de ces derniers à tout le document d'un seul coup en spécifiant des options dans le préambule.

\end{document}