\documentclass[Master.tex]{subfiles}

\begin{document}

Voici un exemple simple montrant quelques fonctionalités de base dans Latex utilisant la classe de document fourni par Elsevier. L'utilisation de Latex peut permettre de faciliter le processus de soumission des articles à des revues scientifiques. Par exemple, pour \emph{Journal of Hydrology}, la soumission initiale consiste à soumettre une version pdf de l'article, avec la mise en page spécifiée dans la classe de document fourni par Elsevier. La seconde étape consiste à soumettre une version pdf révisée de l'article, de même qu'une version mettant en évidence les modifications apportées. Cela peut être fait dans Latex grâce à l'outil latexdiff (\url{www.ctan.org/pkg/latexdiff}). La soumission finale consiste tout simplement à produire un fichier \emph{.zip} du dossier Latex incluant les figures et tous les documents Latex. Puisque les images sont déjà spéparés du code dans Latex, cela permet de faciliter et d'accélérer le processus de soumission de la version finale de l'article.

\end{document}