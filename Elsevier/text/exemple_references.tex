\documentclass[exemple_master.tex]{subfiles}

\begin{document}

\section{Les références bibliographiques}

Il faut créer les références dans un fichier \emph{.bib} et référer dans le texte aux petits noms que l'on a donné à chaque. Le style de référence dépendra de ce que l'on va avoir défini dans le préambule. Il y a généralement des formats fournis avec les templates.

On peut faire des références au travers du texte ou entre parenthèses. Par exemple, je peux faire une référence au logiciel \cite{WHAT2016} directement dans le texte ou encore ajouter la référence à la fin entre parenthèse \citep{WHAT2016}. On peut également faire des références multiples aisément, tout est gérer automatiquement à l'interne par Latex \citep{WHAT2016,ladeveze2016}.

\end{document}