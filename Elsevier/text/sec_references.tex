% !TeX spellcheck = fr_CA
\documentclass[../exemple_master.tex]{subfiles}

\begin{document}

\section{Les références bibliographiques}

Les références bibliographiques doivent être sauvegardées dans un fichier \emph{.bib} avec un identifiant unique. Il est alors possible d'ajouter des références bibliographiques dans le texte en se référant directement à leur identifiant unique. Le style des références dans le texte et la bibliographie dépendra de ce qui aura été défini dans le préambule. Il y a généralement des formats fournis avec les classes de document (templates).

Les références peuvent être insérés dans une phrase à travers le texte ou entre parenthèses. Par exemple, il est possible de faire une référence au logiciel \cite{WHAT2016} directement dans le texte ou encore d'ajouter la référence à la fin de la phrase entre parenthèse \citep{WHAT2016}. Il est également possible de faire des références multiples aisément; tout est gérer automatiquement à l'interne par Latex \citep{WHAT2016,ladeveze2016}.

\end{document}