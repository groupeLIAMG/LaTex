
\documentclass[12pt,addrfromleft,addrfromemail,addrfromphone,addrtoleft,orderdatefromto,stdletter,sigleft,dateleft]{newlfm}
\usepackage[frenchb]{babel}
\usepackage[utf8]{inputenc}
\usepackage[T1]{fontenc}
\usepackage{hyperref}
\hypersetup{
    colorlinks=true,
    urlcolor=blue,
}

\urlstyle{same}

% Ajuster les layout des blocks
\newlfmP{dateskipafter=60pt}
\newlfmP{sigsize=10pt}
\newlfmP{Headlinewd=0pt,Footlinewd=0pt} % enleve les ligne horizonatales en hau et en bas du document


% Signature
\namefrom{John Doe}

% Date
\dateset{\today}

% adresse de l'expediteur
\addrfrom{%
    John Doe\\
    490, de la Couronne\\
    Québec G1K9A9\\
    Canada
}
\emailfrom{john.doe@gmail.com} % Email address
\phonefrom{+1 123 456 78 91} % Phone number


% addresse du déstinataire
\addrto{%
    Monsieur Sherlock Holmes\\
    123 Bombay Street\\
    Londres\\
    }

% Salutation
\greetto{Monsieur Holmes,}

% Clôture
\closeline{Veuillez agréer, Monsieur Holmes, l’expression de mes sentiments distingués,}

\begin{document}
\begin{newlfm}
Ceci est un exemple de lettre formelle (lettre de motivation, lettre de
remerciements, \ldots) en utilisant la classe \emph{newlfm.cls}.\\

Cette classe est largement adaptable aux exigences formelles de la lettre pour ce
qui concerne les blocs d'adresses du destinataire et de l'expéditeur ainsi que les formules de salutation et de clôture, la date, \ldots. \\

Pour des détails approfondis sur les options disponibles, vous pouvez vous referez au manuel de la classe (\url{http://www.ctan.org/tex-archive/macros/latex/contrib/newlfm}) ainsi qu’ à cette adresse \url{http://texblog.org/2013/11/11/latexs-alternative-letter-class-newlfm/}.

\end{newlfm}
\end{document}
