\documentclass{article}

\usepackage[left=2cm, right=2cm, top=2cm, bottom=2cm]{geometry}
\usepackage[utf8]{inputenc}

\usepackage{caption,subcaption}
\usepackage{floatrow}
\usepackage{graphicx}

%==== Pour mettre une page en format paysage ====

\usepackage{pdflscape}

%==== Pour les entêtes et pieds de page ====

\usepackage{fancyhdr}
\pagestyle{fancy}
\lhead{Entête générée grâce au module \emph{fancyhdr}}
\rhead{page \thepage}
\cfoot{Page mise au format paysage avec le module \emph{pdflscape}}
\renewcommand{\headrulewidth}{0.4pt}
\renewcommand{\footrulewidth}{0.4pt}

\begin{document}

% Pour que la page en format paysage soit "flottante" dans le document et ne brise pas le flow du texte, on peut utiliser le module "afterpage". Voir exemple "floating_landscape_single_page_figure.tex".

\begin{landscape}
% set horizontal space between subfigures:
\thisfloatsetup{subfloatrowsep=qquad} 
% put "Figure 1" alone on the first line on top of the text:
\captionsetup{labelsep=newline}
\begin{figure}[h]
    \ffigbox
    {}
    {
    \begin{subfloatrow}[3]
        \ffigbox
        {
         \caption{sub-caption of subfigure a}
         \label{subfig1}
        }
        {\includegraphics[width=\linewidth]{./img/67972605}}
        
        \ffigbox
        {
         \caption{sub-caption of subfigure b}
         \label{subfig2}
        }
        {\includegraphics[width=\linewidth]{./img/67972605}}
        
        \ffigbox
        {
         \caption{sub-caption of subfigure c}
         \label{subfig1}
        }
        {\includegraphics[width=\linewidth]{./img/67972605}}
    \end{subfloatrow}
    
    \vspace{1em}
    
    %Il faut décorer l'environnement subfloat ci-dessous avec * pour indiquer que la légende de la figure se retrouve à l'intérieur de ce subfloat. C'est pourquoi il est nécessaire d'utiliser la commande \subcaption du module "subcaption" pour indiquer les légendes des sous-figures à l'intérieur de l'environnement subfloatrow lorsque ce dernier est décoré par un *. En temps normal on pourrait tout simplement utiliser \caption et le subfloatrow s'occuperait du reste, tel que pour le subfloatrow ci-dessus.
    
    \begin{subfloatrow*}[3]    
        \ffigbox
        {
         \subcaption{sub-caption of subfigure d}
         \label{subfig1}
        }
        {\includegraphics[width=\linewidth]{./img/67972605}}
        
        \ffigbox
        {
         \subcaption{sub-caption of subfigure e}
         \label{subfig1}
        }
        {\includegraphics[width=\linewidth]{./img/67972605}}
        
        \ffigbox
        {
         \RawCaption{ 
         \caption{La commande \emph{$\backslash$RawCaption} permet de libérer le contenu de la légende et de la placer à l'endroit désiré sans perturber la mise en page des éléments à l'intérieur de l'environnement ``float''. Dans cet exemple, la légende de la figure est considérée comme une sixième sous-figure dans la mise en page.}
         \label{fig:param_uncertain}}          
        }
        {}
    \end{subfloatrow*}
    }
\end{figure}
\end{landscape}
     
\end{document}